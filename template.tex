%%
%% This is file `template.tex',
%% generated with the docstrip utility.
%%
%% The original source files were:
%%
%% nddiss2e.dtx  (with options: `template')
%% 
%% This is a generated file.
%% 
%%  Copyright (C) 2004-2005 Sameer Vijay
%% 
%%  This file may be distributed and/or modified under the
%%  conditions of the LaTeX Project Public License, either
%%  version 1.2 of this license or (at your option) any later
%%  version. The latest version of this license is in
%%     http://www.latex-project.org/lppl.txt
%% 
%% 
%% ==============================================================
%% 
%% Notre Dame's Dissertation document class by Sameer Vijay
%% that adheres to the University of Notre Dame guidelines
%% published in Spring 2004.
%% 
%% Please send any improvements/suggestions to :
%%     Shari Hill, Graduate Reviewer.
%%     shill2@nd.edu
%% 
%% For documentation on how to use nddiss2e class, process the
%% file nddiss2e.dtx through LaTeX.
%% 
%% ==============================================================
%% 
\ProvidesFile{template.tex}
    [2016/10/16 v3.2016%
     Template file for NDdiss2e class]
\documentclass[draft]{nddiss2e}
                     % One of the options draft, review, final must be chosen.
                     % One of the options textrefs or numrefs should be chosen
                     % to specify if you want numerical or ``author-date''
                     % style citations.
                     % Other available options are:
                     % 10pt/11pt/12pt (available with draft only)
                     % twoadvisors
                     % noinfo (should be used when you compile the final time
                     %         for formal submission)
                     % sort (sorts multiple citations in the order that they're
                     %       listed in the bibliography)
                     % compress (compresses numerical citations, e.g. [1,2,3]
                     %           becomes [1-3]; has no effect when used with
                     %           the textrefs option)
                     % sort&compress (sorts and compresses numerical citations;
                     %           is identical to sort when used with textrefs)
\begin{document}

\frontmatter             % All the items before Chapter 1 go in ``frontmatter''

\title{ Title of Work }  % Title

\author{ Jane Doe }      % Author's name
\work{ Dissertation }    % ``Dissertation'' or ``Thesis''
\degaward{ Doctor of Philosophy }  % Degree you're aiming for.
                                   % Should be one of the following options:
                                   % ``Doctor of Philosophy'' (do NOT include ``in Subject'')
                                   % ``Master of Science \\ in \\ Subject''
\advisor{ John Public }  % Advisor's name
 % \secondadvisor{ }     % Second advisor, if used option ``twoadvisors''
\program{ }              % Name of the graduate program

\maketitle               % The title page is created now

 % You must use either the \makecopyright option or the \makepublicdomain option.
 % \copyrightholder{ }   % If you're not the copyright holder
 % \copyrightyear{ }     % If the copyright is not for the current year
 % \makecopyright        % If not making your work public domain
                         % uncomment out \makecopyright
 % \makepublicdomain     % Uncomment this to make your work public domain

 % Including an abstract is optional for a master's thesis, and required for a
 % doctoral dissertation.
 % \begin{abstract}
 % \end{abstract}
 %                       % Either place the text between begin/end, or
 % \include{abstract}    % put it in a file to be included

 % Including a dedication is optional.
 % \renewcommand{\dedicationname}{\mbox{}} % Replace \mbox{} if you want
                                           % something else.
 % \begin{dedication}
 % \end{dedication}
 %                       % Use one of the two choices to add dedication text
 % \include{dedication}

\tableofcontents
\listoffigures
\listoftables

 % Including a list of symbols is optional.
 %% \renewcommand{\symbolsname}{newsymname} % Replace ``newsymname'' with
                                            % the name you want, and uncomment
 % \begin{symbols}
 % \end{symbols}
 %                       % Use one of the two choices to add symbols text
 % \include{symbols}

 % Including a preface is optional.
 %% \renewcommand{\prefacename}{ } % If you want another Preface name, add
                                   % something else, and uncomment.
 % \begin{preface}
 % \end{preface}
 %                       % Use one of the two choices to add preface text
 % \include{preface}

 % Including an acknowledgements section may or may not be optional. It's hard to
 % tell from the information available in Spring 2013.
 %% \renewcommand{\acknowledgename}{ } % If you want another Acknowledgement name
                                       % add something else, and uncomment
 % \begin{acknowledge}
 % \end{acknowledge}
 %                       % Use one of the two choices to add acknowledge text
 % \include{acknowledgement}

\mainmatter
 % Place the text body here.
 % \include{chapter-one}
 % Begin each chapter with \chapter{Title}.

\appendix

 % If you have appendices, add them here.
 % Begin each one with \chapter{TITLE} as before. The \appendix command takes
 % care of renaming chapter headings and creates a new page in the Table of
 % Contents for them.
 % \include{appendix-one}

\backmatter              % Place for bibliography and index


\bibliographystyle{nddiss2e}
 \bibliography{ }           % input the bib-database file name


\end{document}

%%
\endinput
%%
%% End of file `template.tex'.
