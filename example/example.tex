%
% Modified by Megan Patnott
% Last Change: Jan 18, 2013
%
%%%%%%%%%%%%%%%%%%%%%%%%%%%%%%%%%%%%%%%%%%%%%%%%%%%%%%%%%%%%%%%%%%%%%%%%
%
% Modified version of the sample_ndthesis.tex
% by Sameer Vijay
% Last Change: Wed Jul 27 2005 14:00 CEST
%
%%%%%%%%%%%%%%%%%%%%%%%%%%%%%%%%%%%%%%%%%%%%%%%%%%%%%%%%%%%%%%%%%%%%%%%%
%
% Sample Notre Dame Thesis/Dissertation
% Using Donald Peterson's ndthesis classfile
%
% Written by Jeff Squyres and Don Peterson
%
% Provided by the Information Technology Committee of
%   the Graduate Student Union
%   http://www.gsu.nd.edu/
%
% Nothing in this document is serious except the format.  :-)
%
%%%%%%%%%%%%%%%%%%%%%%%%%%%%%%%%%%%%%%%%%%%%%%%%%%%%%%%%%%%%%%%%%%%%%%%%
% This is *not* a substitute for the documentation, which is included
% as a pdf file in the standard distribution, and can be obatined from
% the dtx file in the advanced distribution.
%%%%%%%%%%%%%%%%%%%%%%%%%%%%%%%%%%%%%%%%%%%%%%%%%%%%%%%%%%%%%%%%%%%%%%%%
%
% You should *also* have a ND formatting guide to ensure that you have
% all the relevant parts, put the captions in the right place, etc.
% Just because you have this wonderful style classfile doesn't mean
% that it removes *all* the formatting onus from you.  :-)
% Although be warned that the Graduate School has been known to let
% their official formatting guide get out of date. When in doubt,
% the Microsoft Word example seemed to be the only thing kept
% consistently up-to-date in 2013, and is probably the safest thing
% to consult.
%
% You should break all of this stuff up into separate files
% (at the very least, one chapter per file) and use the \include
% command, as has been done here for chapters 1 and 2 and the appendix.
% There is also an \input command, but \include is more commonly used to
% import chapters in books and dissertations. For the differences between these
% two commands, see, e.g., 
% http://web.science.mq.edu.au/~rdale/resources/writingnotes/latexstruct.html
% or http://tex.stackexchange.com/questions/246/when-should-i-use-input-vs-include.
%
% If you compile from the command line, note that you should also have 
% a good Makefile; one that invokes LaTeX as many times as necessary 
% (up to 4) and bibtex if necessary.
%
% If you use an editor that allows you to compile from within the
% program, note that you will need to compile up to four times. Also,
% we recommend that you use pdflatex (sometimes displayed as
% LaTeX => PDF) to compile directly to pdf.
%
% If you have any suggestions, comments, questions, please send e-mail
% to: dteditor@nd.edu
%
%%%%%%%%%%%%%%%%%%%%%%%%%%%%%%%%%%%%%%%%%%%%%%%%%%%%%%%%%%%%%%%%%%%%%%%%

\documentclass[final,numrefs,sort&compress,twoadvisors]{nddiss2e}
% One of the options draft, review, final must be chosen.
% One of the options textrefs or numrefs should be chosen
% to specify if you want numerical or ``author-date''
% style citations.
% Other available options are:
% 10pt/11pt/12pt (available with draft only)
% twoadvisors
% noinfo (should be used when you compile the final time
%         for formal submission)
% sort (sorts multiple citations in the order that they're
%       listed in the bibliography)
% compress (compresses numerical citations, e.g. [1,2,3]
%           becomes [1-3]; has no effect when used with
%           the textrefs option)
% sort&compress (sorts and compresses numerical citations;
%           is identical to sort when used with textrefs)

\begin{document}

\frontmatter % All the items before the first chapter go in ``frontmatter''

% Titles may be 1-4 lines long. If your title is longer than 4 lines,
% the class file may have difficulty formatting the title page.
% Line-breaks in the title have to be protected with `\protect`.
\title{Gnus and you \protect\\ a Brief \protect\\ on all and
Everything \protect\\ About Gnus in our Society}
\author{Gerald G. Gnastich}
\work{Dissertation} % or \work{Thesis}
%\degaward{Doctor of Philosophy} % or 
\degaward{Master of Science \\ in \\ Subject}
\advisor{Gary Greenfield}
\secondadvisor{Gordon Gray} % if you have two advisers are using the option twoadvisors
\department{Gnulogy}

\maketitle
%%%%%%%%%%%%%%%%%%%%%%%%%%%%%%%%%%%%%%%%%%%%%%%%%%%%%%%%%%%%%%%%%%%%%%%%
%
% Front stuff
%
%%%%%%%%%%%%%%%%%%%%%%%%%%%%%%%%%%%%%%%%%%%%%%%%%%%%%%%%%%%%%%%%%%%%%%%%

% You must either set the copyright information or put your work in the public domain.
\copyrightholder{Garry Greene} % See template or documentation for
\copyrightyear{2005}           % other copyright options.
\copyrightlicense{CC-BY-4.0}
\makecopyright

% An abstract is optional for a mster's thesis, and required for a doctoral dissertation.
\begin{abstract}
  Please note that the full \LaTeX\ source code (and an associated
  \texttt{Makefile}) is available from the University of Notre Dame
  Graduate Student Union web site.  The Information Technology
  Committee page\footnote{\url{http://www.gsu.nd.edu/}}
  has all the necessary files in download-able form.  This particular
  dissertation was developed under Unix, but is also be usable
  under Windows with the appropriate \LaTeX\ setup and was modified
	on a Windows system in 2012-2013. It should also work with on Mac.
  
  While the source code for this document provides an excellent
  example for how to use the \nddiss\ \LaTeX\ class to write a
  Notre Dame thesis, it is \emph{not} a substitution for the
  documentation of the \nddiss\ \LaTeX\ class (also available on
  the ND GSU web site).

  In this thesis, I will tell all that I know about Gnus.  Gnus are
  wonderful little creatures that inhabit the center of the earth and
  give us wonderful and plentiful trees, dirt, and other
  earthly-things.
  
  In short, we should love and cherish the Gnus.  They can be very
  friendly, and are often mistaken for squirrels on the University of
  Notre Dame campus.  Feed them whenever possible.  If they get caught
  in trash cans, tip them over so that they can get out.

  This abstract is going to continue on, including a few formulas,
  just for the sake of spilling over on to two pages so that we can
  see the author's name in the top right corner:
  
	\begin{align*}
    a^2 + b^2 &= c^2 \\
    E &= mc^2 \\
    \frac{e}{m} &= c^2 \\
    a^2 + b^2 &=\frac{e}{m}
  \end{align*}

  These equations, by themselves mean nothing.  But to the common Gnu,
  they define a whole way of living.  While intricate mathematical
  implications certainly do not infiltrate the majority of humans'
  lives, every Gnu, from birth, is imbued with a sense of mathematical
  certainty and guidance.  All Gnus, great and small, feel at one with
  mathematics.  The cute furry bit is just a scam for their
  calculating minds.
\end{abstract}

% A dedication is optional.
\renewcommand{\dedicationname}{NEW DEDICATION NAME}

\begin{dedication}
  To George, my favorite Gnu
\end{dedication}

% These are required, and must be in this order.
\tableofcontents
\listoffigures
\listoftables

% A preface is optional.
\begin{preface}
  I would like to preface this work with all the wonderful things that
  Gnus have brought to our society: trees, dirt, flowers, grass,
  lakes, and other earthly-things.  We should not forget them in our
  daily lives.

  Additionally, we should offer them food for all their hard work.  In
  fact, Gnus work so hard that they sleep for the colder half of
  the year.  As such, they tend to grow a little rotund.  Humans
  should not fault them for this, as it is necessary for their
  survival.  Indeed, many humans grow rotund on their on accord!
\end{preface}

% It's hard to tell from the information available from the Graduate
% School in Spring 2013 whether or not an acknowledgements section is optional.
\begin{acknowledge}
  I would like to acknowledge all the loving Gnus at Notre Dame.
  Particularly the one that comes to the window in the Hayes Healy
  building.  He (she?) has given me much inspiration, love, and dirt.
  I would also like to thank my advisor, Dr.\ Gary Greenfield, with
  whom this work would not have been possible.

  Finally, I would like to thank the U.S.\ Government, Department of
  Gnus, for their generous grant, number GNU3042920920.3, which
  allowed me to pursue my work.
\end{acknowledge}

% A symbols section is optional.
\begin{symbols}
  \sym{\mathcal{F}}{sighting frequency of Gnus about campus}
  \sym{p}{student population}
  \sym{f}{type of food available}
  \sym{d}{day of week}
  \sym{c}{speed of light}
  \sym{m}{mass}
  \sym{e}{elementary charge}
  \sym{a,b}{miscellaneous constants}  
  \sym{E}{energy}  
\end{symbols}

\mainmatter
% Place the text body here.
%\include{chapter-one}
%Begin each chapter with \chapter{Title}. Both the thesis title and
%chapter titles should match in style.

%
% An unnumbered chapter (features)
%
\unnumchapter{Features of Formatting in This Example File}
% The \unnumchapter command allows you to include an unnumbered chapter as part of
% the main text before Chapter 1. It will appear in your table of contents, and you
% should have at most one such chapter (although nothing in the class file will
% prevent you from creating more).

% The usual \cite{} command is also available, and should work as expected.
This \verb+chapter+ has been added to the original sample file to highlight the
various features with the formatting that conforms to the Graduate school
guidelines --- whether obtained due to the use of \nddiss\/ class file or just
plain good practice.
\begin{itemize}
\item An important note on line-breaks via \verb+\\+ in titles: the
  titles of the thesis as well as chapters and table captions use
  \verb+\MakeTextUppercase{}+ from the \verb+textcase+ package.  Due
  to the nature of the \verb+center+ environment, any line-breaks
  introduced in titles and captions should be protected, as in
  \verb+\protect\\+.
  To preserve the case in titles and captions, use, e.g.,
  \verb+\NoCaseChange{Gnus}+.
\item In the \emph{dedication}, the title name has been modified. So, you know
how to and that it can be done.
\item The entries in the \emph{List of figures} and \emph{List of Tables} are
single-spaced themselves but are double-spaced from the other.
\item The table captions are not in all CAPS as well for the reason mentioned
above.
\item Appropriate space is left between the \verb+Table xx+ and its
corresponding caption (which is double-spaced itself) as in table \ref{tbl:bogus1}.
\item Tables look much better without the vertical lines (good practice).
\item There is double-spacing between the table entries but single-spacing
within the entry.
\item The chapter (see Chapter \ref{chap:golfing}) or section titles are
double-spaced as mentioned in the guidelines.
\item There is a \verb+subsubsection+ present (eg. section \ref{sec:data}) and
is properly formatted in the TOC.
\item Sections deeper than \verb+subsubsection+ should not appear in the TOC.
\item Table \ref{tbl:defs} is an example of the use of \textsf{landscape}
environment in which a normal table is formatted in a \emph{landscape} mode.
\item The \textsf{longtable} environment is used in Tables \ref{tbl:votes} and
\ref{tbl:rotated-rankings}, in normal and \verb+landscape+ mode, respectively. The
table captions are formatted properly in both cases.
\item In the table \ref{tbl:votes}, the \verb+footnote+ in the table header 
does not appear at all. This is not an error of the \nddiss\/ class but of the
\textsf{longtable} package.
\item An example of citing a website is shown in the bibliography (see
\citep{gairley2000}) which is formatted using the \verb+nddiss2e.bst+
citation style file.
\item A bit of information on the \nddiss\/ class file and the typesetting program
used is included in a box on the last page of the thesis.
\item Footnotes should space properly.
\item Items in \verb+itemize+, \verb+enumerate+, and \verb+description+ environment
should automatically single-space within an item, but double space between items.
\end{itemize}

%
% Chapter 1
%

\include{chapter1}


%
% Chapter 2
%

\include{chapter2}


%
% Appendix (optional)
%

\appendix

\include{appendix}


%
% Back stuff
%

% % comment out the following three lines
% if using chapter-wise bibliography

 \backmatter
 \bibliographystyle{abbrvnat} % The standard abbrvnat style should be acceptable. Also provided with both the advanced and standard
 \bibliography{example}       % distributions are nddiss2e, nddiss2enoarticletitles, and nddiss2enosort style options.
% If you prefer to manually enter your bibliography, that is fine. Comment out the previous two lines, and enter your bibliography
% as usual. Note that if you choose this route, formatting the bibliography is your responsibility. An example is below, including the
% optional arguments necessary for author-date style citations.
%	\begin{thebibliography}{9}
%		\bibitem[Galmira(1998)]{galmira98:_gnus_milit} G.\ Galmira. Gnus and the military -- a secret conspiracy? \emph{Growing Towards Gnu}, III(7):22--183, September 1998.
%		
%		\bibitem[Ganston and Greenfield(1998)]{gnus98:_gerry_ganst} G.\ Ganston and G.\ Greenfield. \emph{Gnus and You: The Art of Being New}. volume I. Grapping Books, NY, August, 1998.
%		
%		\bibitem[Gloonston(1998)]{gloonston98:_gnuly_discov_gnus} G.\ Gloonston. Newly discovered gnus: The LoG. \emph{Growing Towards Gnu}, II(12):23---57, March 1998.
%		
%		\bibitem[Greenfield(1996)]{greenfield96:_gettin_know_gnu} G.\ Greenfield. \emph{Getting to Know Gnu}. PhD thesis, Geoffrey Garfield School of Gnus, August 1996.
%		
%		\bibitem[van Gairley(2000)]{gairley2000} G.\ van Gairley. Gnu's review. Website, 2000. \url{http://www.gairley.gnu}.
%	\end{thebibliography}

\end{document}

% End of ``example.tex''
